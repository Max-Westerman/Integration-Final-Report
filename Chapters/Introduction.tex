\documentclass[11pt]{report}
\usepackage{StyleSheets/main}
\begin{document}


\chapter{Introduction}\label{ch:introduction}

This project report goes into depth about a vehicle (also referred to as a system) that has the ability to complete an obstacle course test as well as a pickup and placement test in a multitude of ways. On the obstacle course, outlined by white lines on the side and green tape on the front and back end, our vehicle will be able to stay within the white lines, detect the walls within the course, maneuver around them, and stop on the green line at the end of the course. For the pickup and placement course, the vehicle will be able to detect an I-beam, pick up a colored box, and, based on its color and size, drop the box off on the other end of the course and return to the starting side of the court. Completing these tests requires communication skills, an array of engineering skills such as computer and electrical skills, writing skills, and decision-making skills. The project intends to strengthen all of the required skills needed to solve a complex problem that meets the demand being asked.
\par The purpose of the report is to detail the process of creating a system that will meet all requirements. The report was also a great way to get the team’s ideas flushed out and catch mistakes within our project that we could then correct. \Cref{ch:project-management-plan} keeps a timeline of the system’s progress by giving weekly updates that include the team member(s) who is responsible for the task and the duration of the task. Another key part of the report is \Cref{ch:system-requirements}.  In this part of the report, the system requirements for certain objectives such as the obstacle avoidance and pickup and placement of the vehicle are established. Other major chapters of the report are \cref{ch:conceptual-design,ch:detailed-design}. \Cref{ch:conceptual-design} outlines the conceptual designs that the team came up with, and \cref{ch:detailed-design} details the design of the system, which includes algorithms and other electrical and computer engineering tools that were used in the project. Lastly, \cref{ch:system-test} goes into depth about \gls{TEMPS}. In this chapter, an explanation of how the system requirements are tested is provided.

\end{document}