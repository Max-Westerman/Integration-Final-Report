\documentclass[11pt]{report}
\usepackage{StyleSheets/main}
\begin{document}


% ADD TO TABLE OF CONTENTS
\chapter*{Executive Summary}
BioComp, LLC, has asked our class of students to compete for a contract to manufacture and maintain a robot meant for work within the company's cleanroom. The robot is to be autonomous and capable of aiding in the organization of the cleanroom. The robot must be able to complete two tasks: an object pickup-and-place sequence, and an obstacle avoidance sequence. These two tasks test the vehicles ability to navigate the cleanroom floor with ease, increasing the efficiency of the workspace. This robot was designed over the course of twenty weeks, with the first ten dedicated to studying the available components and relevant coding, the second dedicated towards the design and manufacturing of the robot.
\par The final design, see \cref{fig:front-view}, was developed from a concept generated by redesigning the foundational model with varying levels of added complexity, a process that provided foresight for potential complications during development. The final iteration consists of a squared omni-directional wheel configuration, three ultrasonic sensors to detect distances from various points along the robot's front edge, three color sensors used for movement and 1 used for object color detection, one infrared sensor array with eight unique sensors for line following optimization, and one button used for object size detection. The robot was able to complete all tasks. The added features which aided in the robot's success include the infrared sensor array, 3D-printed sensor mounts for optimized sensor placement, suspension system to maintain proper traction on the cleanroom floor, and gearbox modifications for significantly increased speed.
\par The \cref{ch:project-management-plan} closely follows the time dedicated by each team member on work days. The team made only a few generalized timeline-specific goals knowing that complications would occur. Additionally, the modular nature of our design meant that one aspect of the project could work forwards even if another subsystem hit a stuck point.
\par The purpose of this project was to expose engineering students to the design process of building an autonomous system, and provide experience with (simulated) client-engineer communication. The team had set high expectations at the beginning of the project and went to great lengths to ensure they were achieved, successfully completing all objectives presented.

\chapter{Introduction}\label{ch:introduction}

This project report goes into depth about a vehicle (also referred to as a system) that has the ability to complete an obstacle course test, as well as a pickup and placement test in a multitude of ways. On the obstacle course, outlined by white lines on the side and green tape on the front and back end, our vehicle will be able to stay within the white lines, detect the walls within the course, maneuver around them, and stop on the green line at the end of the course. For the pickup and placement course, the vehicle will be able to detect an I-beam, pick up a colored box, and, based on its color and size, drop the box off on the other end of the course and return to the starting side of the court. Completing these tests requires communication skills, an array of engineering skills such as computer and electrical skills, writing skills, and decision-making skills. The project intends to strengthen all of the required skills needed to solve a complex problem that meets the demand being asked.
\par The purpose of the report is to detail the process of creating a system that will meet all requirements. The report was also a great way to get the team’s ideas flushed out and catch mistakes within our project that could then be corrected. \Cref{ch:project-management-plan} keeps a timeline of the system’s progress by giving weekly updates that include the team member(s) who is responsible for the task and the duration of the task. Another key part of the report is \Cref{ch:system-requirements}.  In this part of the report, the system requirements for certain objectives such as the obstacle avoidance and pickup and placement of the vehicle are established. Other major chapters of the report are \cref{ch:conceptual-design,ch:detailed-design}. \Cref{ch:conceptual-design} outlines the conceptual designs that the team came up with, and \cref{ch:detailed-design} details the design of the system, which includes algorithms and other electrical and computer engineering tools that were used in the project. Lastly, \cref{ch:system-test} goes into depth about \gls{TEMPS}, where an explanation of how the system requirements are tested is provided.

\end{document}