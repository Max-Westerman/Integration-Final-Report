\documentclass[11pt]{report}
\usepackage{StyleSheets/main}
\begin{document}

\chapter{System Analysis \& System Performance Metrics}\label{ch:system-analysis}
This section details the goals for the project and compares them to the performance of the final system. The system will be analyzed and the criteria for the objectives will be set and measured. 

\section{Max Power Utilization Measurement}
This vehicle could run at full power for much longer than what was required for these tests which took place. The vehicle never ran low on battery except for after multiple days of testing and experimenting, often over 5 hours a day. The power utilization of this vehicle allowed for prolonged use of the robot without any performance dips or irregular malfunctions. 
\section{Task 1: Obstacle Avoidance}
The vehicle performed excellently in this test. The criteria for this objective was that the robot will be placed on an obstacle course and will need to avoid the boundaries of the course as well as avoid the walls placed in front of it. This objective was successfully completed in an impressive time, being one of the top times of all teams who tested. This was deemed to be a success for the team. 
\par The robot did not at any time during these tests do anything which can be deemed as a failure. The robot did not cross the white boundary line at any point, and did not come into contact with any of the walls. The color sensors and ultrasonic sensors on the vehicle did their job in protecting the robot from making any mistakes under this category.
\section{Task 2: Pickup and Placement}
The second task for this project proved to be the tougher of the 2 to complete, however this vehicle completed the tasks appropriately and efficiently. After much testing, the vehicle completed the pickup and placement test 26 seconds, over 2x faster than any other group. As the robot moves faster the logic behind the robot exponentially increases, as a much slower robot could deal with slightly removing themselves from the track as they would still be within the margin. A 2x speed improvement signifies a much larger improvement in the design and implementation of the system.

At no point in the pickup and placement tests did the vehicle lose sight of the line being followed, cross any boundaries considered the edge of the course, miss or poorly pickup the box, or make any incorrect decisions regarding the box size and color and which line was correct to be followed. 
\section{Performance vs Other Teams}
The vehicle performed exceptionally well when compared to other teams vehicles. Not only being one of the first few to finish both of the necessary tests, this vehicle was a top performer in both categories. The system was impressive in both efficiency and speed. 

\section{Specs}
\begin{description}
    \item[Weight:] \SI{2.59}{\kilogram}
    \item[Wheel-Base Dimensions (W x L):] \SI{232}{mm} by \SI{309}{mm}
    \item[Battery Run Time:] Approximately 3 hours
    \item[Task 1 Completion time:] 26 seconds
    \item[Task 2 Completion time:] 40 seconds
\end{description}

\end{document}